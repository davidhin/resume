\documentclass[10pt,a4paper,ragged2e,withhyper]{altacv2}
\geometry{left=1.4cm,right=1.4cm,top=2cm,bottom=2cm,columnsep=1cm}
\usepackage{paracol}
\usepackage[default]{lato}

% Change the colours if you want to
\definecolor{VividPurple}{HTML}{000000}
\definecolor{SlateGrey}{HTML}{2E2E2E}
\definecolor{LightGrey}{HTML}{555555}
\definecolor{PaleGrey}{HTML}{777777}
% \colorlet{name}{black}
\colorlet{tagline}{black}
\colorlet{heading}{black}
\colorlet{headingrule}{black}
% \colorlet{subheading}{PastelRed}
\colorlet{accent}{black}
\colorlet{emphasis}{SlateGrey}
\colorlet{body}{LightGrey}
\colorlet{palegrey}{PaleGrey}

% Change the bullets for itemize and rating marker
% for \cvskill if you want to
\renewcommand{\itemmarker}{{\small\textbullet}}
\renewcommand{\ratingmarker}{\faCircle}

\begin{document}
\name{David Hin}
\personalinfo{%
  \email{hello@davidhin.com}
  \phone{+61424875767}
  \linkedin{davidhin}
  \github{davidhin}
}
\makecvheader
\medskip\medskip

\tolerance=9999
\emergencystretch=10pt
\hyphenpenalty=10000
\exhyphenpenalty=100

%% Depending on your tastes, you may want to make fonts of itemize environments slightly smaller
\AtBeginEnvironment{itemize}{\small}

%% Set the left/right column width ratio to 6:4.
\columnratio{0.64}

% Start a 2-column paracol. Both the left and right columns will automatically
% break across pages if things get too long.
\begin{paracol}{2}

% ---------------------------------------------------------------- %
%                            EXPERIENCE                            %
% ---------------------------------------------------------------- %

\cvsection{Experience}

\cveventalt{Centre for Research on Software Engineering \\Technologies}{Development of an automated tool for detecting C/C++\\ software vulnerabilities}{Mar 2021 -- Oct 2021}{Python, Docker, PyTorch / PyTorch Lightning}
\begin{itemize}
\item Conceptualised and implemented an automated software vulnerability detection tool for C/C++ code focused on usability and explainability. The tool leveraged graph neural network architectures to outperform the state-of-the-art model by 104\%.
\item \boldline{Hin, D.}, Kan, A., Chen, H. and Babar, M.A., 2021. LineVD: Statement-level Vulnerability Detection using Graph Neural Networks. \textit{Under Review}.
\end{itemize}

\cveventalt{}{Development of an automated pipeline for software vulnerability assessment in Java}{Jun 2020 -- Jan 2021}{Python, Java, Keras/Tensorflow}
\begin{itemize}
\item Designed an automated software vulnerability assessment pipeline for Java code using deep learning, achieving 50\% higher accuracy and requiring 6.3x less time to train compared to baseline models.
\item Le, T.H., \boldline{Hin, D.}, Croft, R. and Babar, M.A., 2021. DeepCVA: Automated Commit-level Vulnerability Assessment with Deep Multi-task Learning. In \textit{Proceedings of the 36th IEEE/ACM International Conference on Automated Software Engineering} (CORE A*)
\end{itemize}

\cveventalt{}{Analysis and visualisation of online security \\vulnerability discussions}{Jan 2020 -- Mar 2020}{Python, SQL, Scikit-Learn, Docker}
\begin{itemize}
\item Processed >50GB of data from StackExchange and applied topic modelling to get insights on the key types of vulnerability discussion topics being discussed by developers
\item Le, T.H., Croft, R., \boldline{Hin, D.} and Ali Babar, M.A., 2021. A Large-scale Study of Security Vulnerability Support on Developer Q\&A Websites. In \textit{Evaluation and Assessment in Software Engineering} (CORE A)
\end{itemize}

\cveventalt{}{Development of security-related text classifier from online discussions}{Oct 2019 -- Jan 2020}{Python, Scikit-Learn, Docker}
\begin{itemize}
\item Used machine learning, deep learning, and natural language processing techniques to classify and interpret real-world security-related posts from StackOverflow and the StackExchange network, constructing the largest publicly-available dataset for online security-related discussions.
\item Awarded \$10,000 by the Cybersecurity Cooperative Research Centre.
\item Le, T.H., \boldline{Hin, D.}, Croft, R. and Babar, M.A., 2020. PUMiner: Mining Security Posts from Developer Question and Answer Websites with PU Learning. In \textit{2020 IEEE/ACM 17th International Conference on Mining Software Repositories (MSR). IEEE.} (CORE A).
\end{itemize}

\newpage

\cveventalt{The University of Adelaide}{Supervision of summer projects}{Nov 2020 -- Feb 2021}{Python, React, Express, Node.js, Google Cloud Platform}
\begin{itemize}
\item Conceptualised and oversaw two 12-week summer projects based on web application development, web scraping, and deployment of machine learning technologies.
\item Final deliverables completed on-time and received great feedback from both the students and funder.
\end{itemize}

\cveventalt{}{Development of web application for tracking of vulnerable software components}{Mar 2020 -- Nov 2020}{React, Node.js, Express, MongoDB, Google Cloud Platform, Docker, HTML5/CSS3}
\begin{itemize}
\item Built and deployed a microservice-based backend and front-end web application for innovatively tracking and analysing vulnerable software components.
\item Awarded \$10,000 by the Cybersecurity Cooperative Research Centre.
\item This project achieved the highest mark for final year Honours project for Bachelor of Engineering (Software) at the University of Adelaide (2020).
\end{itemize}

\cveventalt{}{Exploration of deep style transfer in images}{Dec 2018 -- Jan 2019}{Python, Matlab}
\begin{itemize}
\item Leveraged covariance matrix adaptation evolution to transfer high-level image and style features from one image to another, using Python/Keras and Matlab.
\item Work done under Adelaide Summer Research Scholarship (ASRS) program (awarded \$1,200).
\item Alexander, B., \boldline{Hin, D.}, Neumann, A. and Ull-Karim, S., 2019, December. \textit{Evolving pictures in image transition space. In International Conference on Neural Information Processing (pp. 679-690). Springer, Cham.} (CORE B)
\end{itemize}

\divider

\cveventalt{Centre for Nanoscale BioPhotonics}{Development of interactive visualisation-based application for spectroscopy analysis}{Jul 2018 -- Feb 2019}{R, R Shiny}
\begin{itemize}
\item Created an app to analyse data collected from custom built biological autofluorescence spectroscopy equipment, involving manipulation of raw signal data, unsupervised learning, and dimension reduction techniques.
\end{itemize}

\switchcolumn

\cvsection{Summary}
I am an experienced software engineer. My projects address real-world problems across areas as diverse as software security, biology, and computer vision. They demonstrate my expertise in software development, encompassing web, mobile, and desktop-based applications. Many of these projects have been published in peer-reviewed literature, highlighting my ability to see projects to completion while working with different teams. I am seeking an opportunity where I can continually improve my engineering practices and find new ways to optimise complex tasks.

% ---------------------------------------------------------------- %
%                       SKILLS AND COMPETENCIES                    %
% ---------------------------------------------------------------- %
\cvsection{Skills \& Competencies}

\smallsubsection{Software Development}
\cvtag{JavaScript}\cvtag{Node.js}\cvtag{React}\\
\cvtag{HTML}\cvtag{CSS}\cvtag{Git}\\
\cvtag{Firebase}
\cvtag{Amazon Web Services}\\
\cvtag{Google Cloud Platform}
\medskip

\smallsubsection{Data Science}
\cvtag{Pandas}
\cvtag{Scikit-learn}\cvtag{R}
\\\cvtag{Deep Learning}\cvtag{PyTorch/Tensorflow}
\cvtag{High-performance computing (HPC)}
\smallskip

\smallsubsection{Soft skills}
\cvtag{Cross-team collaboration}\\
\cvtag{Mentoring}
\cvtag{Project management}\\
\cvtag{Presentation skills}

% ---------------------------------------------------------------- %
%                             EDUCATION                            %
% ---------------------------------------------------------------- %
\cvsection{Education}
\cvevent{Bachelor of Engineering (Software) with First-class Honours}{University of Adelaide}{Feb 2017 -- Dec 2020 | GPA: 6.8/7}{}


\newpage
% ---------------------------------------------------------------- %
%                            ACHIEVEMENTS                          %
% ---------------------------------------------------------------- %
\cvsection{Achievements}
\begin{itemize}
  \item \yearbold{2021} - Cybersecurity CRC PhD Top-Up Scholarship
  \item \yearbold{2020} - Lifelenz Prize, for achieving the highest Honours mark in Bachelor of Engineering (Software)
  \item \yearbold{2020} - Cybersecurity Cooperative Research Centre Honours Scholarship
  \item \yearbold{2020} - University of Adelaide: Executive\\ Dean’s Award for Academic Excellence
  \item \yearbold{2019} - Cybersecurity Cooperative Research Centre Summer Scholarship
  \item \yearbold{2018} - Australian Oracle User Group Prize, for achieving highest score in Web and Database Computing course
  \item \yearbold{2018} - University of Adelaide: Executive\\ Dean’s Award for Academic Excellence
  \item \yearbold{2017} - University of Adelaide: Executive\\ Dean’s Award for Academic Excellence
  \item \yearbold{2016} - Dux of The Heights School, SA
  \item \yearbold{2015} - Govhack: International Digital Humanities Hack, N3xGen South Australian Champion
\end{itemize}

\end{paracol}
\end{document}
